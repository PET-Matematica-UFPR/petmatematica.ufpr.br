\documentclass[12pt,a4paper]{article}

%----------------------------------------------------
% as duas linhas abaixo devem garantir a acentuação em português, caso não compile o PDF,
% tente comentar/apagar a linha \usepackage[utf8]{inputenc} ou então use os pacotes de
% idiomas que você está acostumado (como latin1 e brazil).
\usepackage[utf8]{inputenc}
\usepackage[portuguese]{babel}
%----------------------------------------------------

\usepackage{amsmath,amsfonts,amssymb}
\usepackage{lmodern}
\renewcommand{\rmdefault}{phv} % Arial
\renewcommand{\sfdefault}{phv} % Arial
\usepackage[left=2.5cm,right=2.5cm,top=2.5cm,bottom=2.5cm]{geometry}
\usepackage{authblk}
\usepackage{sectsty}
\usepackage{nopageno}
\setlength{\affilsep}{4mm}

% define os separadores entre cada autor
\renewcommand\Authsep{ \\ [2mm]}
\renewcommand\Authand{ \\ [2mm]}
\renewcommand\Authands{ \\ [2mm]}

% titulo
\title{T\'itulo do resumo \\ Uma ou mais linhas}

% autor com afiliação 1
\author[1]{
    Nome Sobrenome\thanks{Bolsista }\\ \texttt{email.1@dominio.com}
}

% autores com afiliação 2

\author[2]{
    Nome Sobrenome\thanks{Voluntario} \\
    \texttt{email.2@dominio.com}
}
\author[2]{
    Nome Sobrenome\thanks{Bolsista }\\
    \texttt{email.3@dominio.com}
}

% autores com afiliação 3

\author[3]{
    Nome Sobrenome (Orientador(a))\\
    \texttt{email.4@dominio.com}
}

\author[3]{
    Nome Sobrenome (Corientador(a))\\
    \texttt{email.5@dominio.com}
}

\affil[1]{Instituição 1}
\affil[2]{Instituição 2}
\affil[3]{Instituição 3}

\date{}



\begin{document}
%----------------------------------------------------
%Formatação das referências
\renewcommand{\refname}{Referências}
\sectionfont{\fontsize{12}{12}\selectfont}
%----------------------------------------------------
\maketitle

\noindent {\bf Palavras-chave}: palavra 1, palavra 2, palavra 3. \\[2mm]


\noindent {\bf Resumo}:

Insira aqui o resumo do seu trabalho. É permitido usar tabelas, gráficos e figuras. O caderno de resumos ficará disponível em formato PDF no site da J3M após o evento.

Fique atento ao número de palavras. O texto do resumo deve ter pelo menos 15 linhas e o trabalho todo (incluindo as referências bibliográficas) não pode ultrapassar o limite de \textbf{três páginas}.

Caso esses limites não sejam respeitados, seu trabalho poderá ser devolvido para adequação ou rejeitado. \\[5mm]


Para as Referências Bibliográficas sugerimos seguir o padrão ABNT conforme exemplos disponíveis no endereço:

\verb;http://www.portal.ufpr.br/tutoriais_normaliza/referencia_exemplo.pdf;

Por exemplo, para livros temos os modelos \cite{este livro} e \cite{adams} abaixo:

\begin{thebibliography}{99}
\bibitem{este livro} SOBRENOME, Nome. \textbf{Título:} subtítulo. Local: Editora, Ano.

\bibitem{adams} ADAMS, F.; LAUDGHLINI, G. \textbf{Uma biografia do universo:} do big-bang à desintegração final. Rio de Janeiro: Zahar, 2001.
\end{thebibliography}


\end{document}
